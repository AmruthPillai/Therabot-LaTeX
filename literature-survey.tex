\chapter{Literature Survey}

\pagebreak

\section{Base Papers}

\subsection{Engineering Base Paper}

\noindent
\textbf{Psychology Predictive Model Research based on Artificial Neural Network}
\textit{\textbf{Author:} Cai Zhongxi}

Understanding the psychological state of mind in college students has become crucial due to the increasing mental health issues. The scenario is such that students succumb to emotional stress, learning pressure, relationship issues and inability to express thoughts, and are trapped in depression. This paper focuses on analyzing the incidents that affect the mental status of students by predicting the correlation between the implicating factors. The neural network is trained using a multilayer perceptron to obtain a back propagated output which needs optimization. The authors have proposed a mathematical model for the use of genetic algorithms to improve the mutation rate of the result, thereby, increasing the learning rate.

\subsection{Medical Base Paper}

\noindent
\textbf{Delivering Cognitive Behavior Therapy to Young Adults With Symptoms of Depression and Anxiety Using a Fully Automated Conversational Agent (Woebot):A Randomized Controlled Trial}
\textit{\textbf{Authors:} Kathleen Kara Fitzpatrick, Alison Darcy and Molly Vierhile}

This paper outlines the role of Cognitive Behavior Therapy (CBT) in the treatment of individuals facing anxiety, depression, obsessive compulsive disorder (OCD), post-traumatic stress disorder (PTSD) and anger problems. The authors have created a chatterbot, Woebot, whose sole purpose is to talk to the user and get inputs about his mood. Patient Health Questionnaire-9 and Generalized Anxiety Disorder-7 are some of the standardized questionnaires that every therapist uses to determine the depression severity and frequency. Woebot uses these measures to better understand the individual.

\pagebreak

\section{Additional Papers found through in-depth research}

\noindent
\textbf{Content-Oriented User Modeling for Personalized Response Ranking in Chatbots}
\textit{\textbf{Authors:} Bingquan Liu , Zhen Xu , Chengjie Sun, Baoxun Wang , Xiaolong Wang, Derek F. Wong , Senior Member and Min Zhang}

Automatic chatbots (also known as chat-agents) have attracted much attention from both researching and industrial fields. Generally, the semantic relevance between users’ queries and the corresponding responses is considered as the essential element for conversation modeling in both generation and ranking based chat systems. This paper aims to address the personalized response ranking task by incorporating user profiles into the conversation model.

\noindent
\textbf{On the Construction of more Human-like Chatbots: Affect and Emotion Analysis of Movie Dialogue Data}
\textit{\textbf{Author:} Rafael E. Banchs}

Affect and emotion are inherent properties of human-human communication and interaction. Recent research interest in chatbots and conversational agents aims at making human-machine interaction more human-like in both behavioral and attitudinal terms. This paper intends to present some baby steps in this direction by analyzing a large dialogue dataset in terms of tonal, affective and emotional bias, with the objective of providing a valuable resource for developing and training data-driven conversational agents with discriminative power across such dimensions.

\noindent
\textbf{NLAST: A natural language assistant for students}
\textit{\textbf{Authors:} Fernando A. Mikic Fonte, Martín Llamas Nistal, Juan C. Burguillo Rial, and Manuel Caeiro Rodríguez}

This paper presents a system that works as an assistant for students in their learning process. The assistant system has two main parts: an Android application and a server platform. The Android application is a chatterbot (i.e., an agent intended to conduct a conversation in natural language with a human being) based on AIML, one of the more successful languages for developing conversational agents. The chatterbot acts as an intermediation agent between a student and the server platform.

\pagebreak

\noindent
\textbf{Sentiment Analysis of Text using Deep Convolution Neural Networks}
\textit{\textbf{Authors:} Anmol Chachra, Pulkit Mehndiratta and Mohit Gupta}

Sentiment analysis has been one of the most researched topics in Machine learning. The roots of sentiment analysis are in studies on public opinion analysis at the start of 20th century, but the outbreak of computer-based sentiment analysis only occurred with the availability of subjective text in Web. The task of generating effective sentence model that captures both syntactic and semantic relations has been the primary goal to make better sentiment analyzers. In this paper, we harness the power of deep convolutional neural networks (DCNN) to model sentences and perform sentiment analysis.

\noindent
\textbf{Big data in psychology: using word embeddings to study theory-of-mind}
\textit{\textbf{Authors:} Michel Généreux, Bryor Snefjella and Marta Maslej}

This paper uses a computational approach to estimate the concreteness of words. After examining its reliability and validity, we apply this approach to study a claim within Psychology, namely, that reading a literary fiction story improves the ability to attribute mental states to others.

\noindent
\textbf{Automated Text Messaging as an Adjunct to Cognitive Behavioral Therapy for Depression:A Clinical Trial}
\textit{\textbf{Authors:} Adrian Aguilera, Emma Bruehlman-Senecal, Orianna Demasi and Patricia Avila}

Cognitive Behavioral Therapy (CBT) for depression is efficacious, but effectiveness is limited when implemented in low-income settings due to engagement difficulties including nonadherence with skill-building homework and early discontinuation of treatment. Automated messaging can be used in clinical settings to increase dosage of depression treatment and encourage sustained engagement with psychotherapy.

\noindent
\textbf{Weighted Word2Vec Based on the Distance of Words}
\textit{\textbf{Authors:} Chia-Yang Chang, Shie-Jue Lee and Chih-Chin Lai}

Word2vec is a novel technique for the study and application of natural language processing(NLP). It trains a word embedding neural network model with a large training corpus. After the model is trained, each word is represented by a vector in the specified vector space. The vectors obtained possess many interesting and useful characteristics that are implicitly embedded with the original words. The idea of word2vec is that there are relations between the words if they appear in the neighborhood.
