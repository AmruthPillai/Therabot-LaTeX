\renewcommand{\abstractname}{Abstract}
\renewcommand{\abstractnamefont}{\Huge\textbf}
\renewcommand{\abstracttextfont}{\normalsize}

\begin{abstract}

\OnehalfSpacing

This study aims to highlight the use cases of a variety of technologies in the field of Artificial Intelligence, Speech Synthesis, Personalization, User Character Segmentation and Automated Text/Speech-based Chatbots. To illustrate the uses, we have chosen a specific field of expertise, mental wellness.

We have built an artificially chatbot that is designed to understand the complexities of it’s user, help the user feel more self-aware and spontaneously check back on the user from time to time. We have also devised a method to extract key likes and dislikes from the user’s messages using NLP techniques.

The goal of Therabot as a product to the consumer is not to overthrow the need for real life therapists, but to simply allow the user to feel more comfortable talking about his/her experiences, thus decreasing the stigma that perpetuates around mental health in our country and around the world. We want to make mental wellness more accessible and educate users on the true meaning of depression and anxiety.

The objective of our research is to show how chatbots powered by artificial intelligence can be not only used for service based QA systems, but also in unique niche markets such as medical health, distress systems, education etc. Using our unique methods of generating the user’s responses and creating a schema for each individual user, we are able to bring more personalization and unique feeling to every user.

\end{abstract}
